\documentclass{article}
\usepackage[utf8]{inputenc}
\usepackage[spanish]{babel}
\usepackage{listings}
\usepackage{graphicx}
\graphicspath{ {images/} }
\usepackage{cite}

\begin{document}

\begin{titlepage}
    \begin{center}
        \vspace*{1cm}
            
        \Huge
        \textbf{Parcial 1 - Calistenia}
            
        \vspace{0.5cm}
        \LARGE
        Informática II
            
        \vspace{1.5cm}
            
        \textbf{Juan Felipe Vásquez Bolívar}
            
        \vfill
            
        \vspace{0.8cm}
            
        \Large
        Despartamento de Ingeniería Electrónica y Telecomunicaciones\\
        Universidad de Antioquia\\
        Medellín\\
        Marzo de 2021
            
    \end{center}
\end{titlepage}


\newpage
\section{Tarea 1 }\label{intro}
A continuación, se describirá las instrucciones que se deben seguir para llevar unos objetos de una posición A a una posición B
\subsection{Instruccion 1}
Utilizando solo una mano, levantar la hoja de papel, sin arrugarla.
\subsection{Instrucción 2}
Utilizando la misma mano, descargar la hoja sobre una superficie, de tal manera que la hoja no quede arrugada y se puedan poner objetos sobre ella
\subsection{Instrucción 3}
Utilizando una mano, levantar las dos tarjetas
\subsection{Instrucción 4}
Utilizando la misma mano, descargar las dos tarjetas sobre la hoja, asegurándose que ninguna parte de las tarjetas quede fuera de la hoja
\subsection{Instrucción 5}
Utilizando una mano, levantar las dos tarjetas. Asegurarse que uno de los dos bordes de menor longitud de las tarjetas quede tocando la hoja. Es importante que las tarjetas no queden inclinadas y ninguna parte quede fuera de la hoja.
\subsection{Instrucción 6}
Sosteniendo las tarjetas con la misma mano, y sin dejarlas caer, ubicar el dedo índice de esta mano, en el centro del borde superior de ambas tarjetas.
\subsection{Instrucción 7}
Utilizando la misma mano y sin dejar caer las tarjetas, ubicar el dedo pulgar en uno de los dos laterales de solo una tarjeta, y el dedo anular en el otro lateral de la misma tarjeta.
\subsection{Instrucción 8}
Utilizando la misma mano. sin retirar los dedos de las tarjetas y sin dejarlas caer, utilice los dedos anular y medio para alejar una tarjeta de la otra, hasta que ambas formen una pirámide que se sostenga sola.

\newpage
Por ultimo, los resultados de esta tarea los pueden encontrar en:

https://youtu.be/plvhDJzj0-k



\end{document}
